\documentclass[a4paper, 14pt]{extarticle}

% Language setting
% Replace `english' with e.g. `spanish' to change the document language
\usepackage[T2A]{fontenc}
\usepackage[utf8]{inputenc}
\usepackage[english, russian]{babel}
\usepackage{indentfirst}
\usepackage{setspace}
\usepackage[nottoc]{tocbibind}
\onehalfspacing
\setlength{\parindent}{1.25cm}

\usepackage[top=2cm,bottom=2cm,left=3cm,right=3cm,marginparwidth=1.75cm]{geometry}

% Useful packages
\usepackage{amsmath}
\usepackage{graphicx}
\usepackage[colorlinks=true, allcolors=black]{hyperref}

\title{%
  Влияние пленки естественных примесей на теплообмен на границе рахдела "вода-воздух". \\
  \large Курсовая работа}
\author{Родыгин Вадим}
\begin{document}

\begin{titlepage}
\newpage
\begin{center}
ФЕДЕРАЛЬНОЕ ГОСУДАРСТВЕННОЕ БЮДЖЕТНОЕ ОБРАЗОВАТЕЛЬНОЕ УЧРЕЖДЕНИЕ ВЫСШЕГО ОБРАЗОВАНИЯ «МОСКОВСКИЙ ГОСУДАРСТВЕННЫЙ УНИВЕРСИТЕТ имени М.В.ЛОМОНОСОВА» \\
\end{center}
%\vspace{4em}
\begin{center}
ФИЗИЧЕСКИЙ ФАКУЛЬТЕТ \\ 
\end{center}
\vspace{4em}
\begin{center}
\Large\textsc{\textbf{Влияние пленки естественных примесей на теплообмен на границе раздела "вода-воздух".}}
\end{center}
\vspace{6em}
\newbox{\lbox}
\savebox{\lbox}{\hbox{Пупкин Иван Иванович}}
\newlength{\maxl}
\setlength{\maxl}{\wd\lbox}
\hfill\parbox{13cm}{
\hspace*{5cm}\hspace*{-5cm}Выполнил:\hfill\hbox to\maxl{Родыгин Вадим Игоревич\hfill}\\
\hspace*{5cm}\hspace*{-5cm} \hfill\hbox to\maxl{студент 404 группы\hfill}\\
\\
\hspace*{5cm}\hspace*{-5cm}Научный руководитель:\hfill\hbox to\maxl{Плаксина Юлия Юрьевна}\\
}
\vspace{\fill}
\begin{center}
Москва \\2024
\end{center}
\end{titlepage}

\newpage
\topskip0pt
\vspace*{6em}
\tableofcontents % это оглавление, которое генерируется автоматически
\vspace*{\fill}
\newpage

%\begin{abstract}
%Your abstract.
%\end{abstract}

\section*{Введение}
\addcontentsline{toc}{section}{Введение}

Уже долгое время в физике сплошных сред изучаются свойства поверхности жидкости: образование поверхностного слоя, теплообмен на поверхности, движение поверхности и др. На протяжении многих лет основной проблемой оставались измерения гидродинамических параметров в тонком слое на границе раздела сред --- использование зондов как в работе \cite{katsaros1977heat} дает неточные результаты, так как вблизи зонда рельеф поверхности и течение меняются, а мы получаем информацию только в окрестности зонда. Тем не менее, в упомянутой выше работе экспериментально были получены профили температур $T(z)$ в зависимости от глубины погружения зонда $z$ при помощи зондов.

Первый неинвазивный метод, позволяющий получить информацию о поверхностном слое жидкости был продемонстрирован в работе \cite{spangenberg1961convective}. Результаты этого эксперимента показали, что поверхность может быть как неподвижной, так и движущейся. Следовательно, в зависимости от изучаемой жидкости при численном моделировании требуются различные граничные условия для получения правильных результатов.

Наличие или отсутствие поверхностного слоя зависит от концентрации примесей в растворе. В данном случае рассматривается вода, в которой обычно довольно большое число ионов. В поверхностном слое концентрация ионов ниже, чем в объеме, что приводит к отличным от объема свойствам \cite{water_ions}. При этом в некоторых жидкостях такой слой не образуется, например, в деионизированной воде.

В данной работе численно моделируется теплообмен жидкости с более холодной окружающей средой, что приводит к появлению неустойчивости в системе, а также в зависимости от граничных условий у жидкости появляется холодная пленка или движущаяся поверхность. Основная цель работы --- получение профилей $T(z)$ при различных граничных условиях и их сравнение с экспериментальными значениями, полученными в \cite{katsaros1977heat}. Кроме того, рассматривается зависимость получаемых результатов от размерности модели. В данном случае представлены результаты двумерного и трехмерного моделирования.

\section{Модель}

В работе моделируептся теплообмен воды с окружающей средой. С поверхности постоянно отводится поток тепла, равный  $-210$ Вт/м$^2$ (такой же какой был получен в \cite{katsaros1977heat}), решается две задачи с разными размерностями: двумерная и трехмерная. В двумерном случае размер сосуда составляет 0.05x0.01 м, в трехмерном - 0.05x0.01x0.01 м. Чтобы получить установившееся состояние в сосуде с боковых стенок и со дна подводится дополнительный поток тепла, уравновешивающий потери на испарение.

Рассматривается 2 различных условия на поверхности: твердая стенка и проскальзывание. Во втором случае также работает конвекция Марангони, что приводит к значительным изменениям в поверхностном слое.

Система уравнений:
	\begin{eqnarray}
		\rho\frac{\partial \mathbf{u}}{\partial t} + \rho(\mathbf{u}\cdot\nabla)\mathbf{u} = \nabla\cdot[-\rho\mathbf{l}+\mathbf{K}] + \mathbf{F} + \rho\mathbf{g} \\
		\frac{\partial\rho}{\partial t} + \nabla\cdot(\rho\mathbf{u})=0 \\
		\mathbf{K}=\mu(\nabla\mathbf{u}+(\nabla\mathbf{u})^\mathrm{T}) - \frac{2}{3}\mu(\nabla\cdot\mathbf{u})\mathbf{l} \\
		p_\mathsf{init}=p+p_\mathsf{hydro} \\
		p_\mathsf{hydro} = \rho_\mathsf{ref}\mathbf{g}\cdot(\mathbf{r}-\mathbf{r}_\mathsf{ref}) \\
		\mathbf{u}=0 \\
		\rho\frac{\partial\mathbf{u}}{\partial t}+\rho (\mathbf{u}\cdot\nabla)\mathbf{u} = \nabla\cdot[-\rho\mathbf{l}+\mathbf{K}] + \mathbf{F} + \rho\mathbf{g} \\
		\text{Случай Марангони } \mathbf{u}\cdot\mathbf{n}=0 \\
		\mathbf{K_n}-(\mathbf{K_n}\cdot\mathbf{n})\mathbf{n}=0 \\ \mathbf{K_n}=\mathbf{Kn} \\
		\rho C_\mathsf{p} \frac{\partial T}{\partial t} + \rho C_\mathsf{p} \mathbf{u}\cdot\nabla T + \nabla\cdot\mathbf{q}=Q+Q_\mathsf{p}+Q_\mathsf{vd} \\
		\mathbf{q}=-k\nabla T \\
		\rho=\frac{p_\mathsf{A}}{R_\mathsf{s} T} \\
		-\mathbf{n}\cdot\mathbf{q}=q_0 \\
		Q_\mathsf{vd}=\tau : \nabla\mathbf{u} \\
		\mathbf{n}\cdot[-\rho\mathbf{l}+\mathbf{K}]=\sigma(\nabla_\mathsf{t}\cdot\mathbf{n})\mathbf{n}-\nabla_\mathsf{t}\sigma \\
		\theta_\mathsf{w}=\frac{\pi}{2}
	\end{eqnarray}

\section{Результаты}

В результате моделирования получены профили зависимости $T(z)$ усредненной температуры от высоты при различных граничных условиях в двумерном и трехмерном случаях. Проведено сравнение с известными экспериментальными результатами \cite{katsaros1977heat} на рисунке \ref{katsaros_compare}.

На рисунке \ref{profiles} представлены все получившиеся в результате моделирования профили. На нем можно заметить сильное отличие двумерного и трехмерного случаев. Однако при наличии конвекции Марангони профиль отклоняется от линейного гораздо быстрее, независимо от размерности. 

\section{Выводы}

Двумерный и трехмерный случай дают разные результаты для температуры вблизи поверхности. Однако общая закономерность при постановке различных граничных условий сохраняется. В случае твердой стенки на поверхности профиль остается линейным на большей глубине, чем в случае проскальзывания. Такое поведение соответствует получаемым экспериментальным результатам. Толщина поверхностного слоя при наличии движения вдоль границы сильно уменьшается. Профиль для

Сравнение граничных условий
Сравнение с катсарос

\bibliographystyle{ieeetr}
\bibliography{bibliography}

\end{document}
